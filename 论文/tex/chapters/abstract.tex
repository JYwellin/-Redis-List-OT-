%%%%%%%%%%%%%%%%%%%%%%%%%%%%%%%%%%%%%%%%%%%%%%%%%%%%%%%%%%%%%%%%%%%%%%%%%%%%%%%
% 论文的中文摘要
\begin{abstract}
	\par 协同编辑系统,可以允许不同地点的用户同时编辑同一份文档。为了获得较快的响应和较高的实用性,系统会在不同的地点或设备进行文档的复制。一个用户可以在某个副本上进行文档的编辑,并将做出的修改异步地传递给其他副本。不必要等待服务器处理完再响应用户操作,本地操作可以立即执行。同时系统必须保证编辑的一致性,即在所有用户完成文档的编辑后,所有的副本内容一致。
	\par 可以设计OT函数,并通过控制算法的调用来保证最终结果的一致性,现在ins,del,set等简单operation的OT函数已经基本实现,本次毕业设计的目标是实现Redis List所支持的14种非阻塞操作的OT函数,并且对实现函数的正确性进行验证。阿里云和RedisLab的团队目前都在对Redis List的操作进行开发,Redis List操作的OT函数实现具有应用前景和商业前途。
	\par 针对上述问题,本文的贡献包括:
	\par 首先介绍了Redis List的相关命令,并将Redis List的相关命令进行基于坐标和操作方式的分类,能够更方便地实现OT函数的设计。
	\par 其次对于所有简化后的命令进行了OT函数的数学公式设计。
	\par 最后介绍了TLA +,并通过TLA+实现了对于所设计OT函数的证明,以及对实验结果的分析。
	% 中文关键词。关键词之间用中文全角分号隔开,末尾无标点符号。
	\keywords{协同编辑;操作转换;Redis 系统;TLA+验证}
\end{abstract}

%%%%%%%%%%%%%%%%%%%%%%%%%%%%%%%%%%%%%%%%%%%%%%%%%%%%%%%%%%%%%%%%%%%%%%%%%%%%%%%
% 论文的英文摘要
\begin{englishabstract}
\par To be filled.
	% 英文关键词。关键词之间用英文半角逗号隔开,末尾无符号。
	\englishkeywords{}
\end{englishabstract}