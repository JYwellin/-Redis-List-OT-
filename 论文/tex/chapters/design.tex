\chapter{Redis List OT 函数设计}
\label{chapter:design}
\section{Redis List API 分类}
\subsection{Redis List API简介}
\par 本文针对Redis系统的List命令进行的,首先对Redis中的列表(List)及其相关命令进行一个简要的介绍。
Redis中的列表是字符串列表,按照插入的顺序进行排序,一个列表可以包含的最多元素为$2^{32}-1(4294967295)$个。

\par 列表相关的基本命令共有17个,其中3个为阻塞性操作(即没有元素时会阻塞列表直至等待到有元素可以执行该命令),即其余14种为非阻塞性操作,即下面列出的14个命令。\\

\begin{itemize}
\item LINDEX  key index:通过索引index来获取列表key中的元素
\item LINSERT key BEFORE(AFTER) pivot value:在列表key某个元素pivot的前面(BEFORE)或者后面(AFTER)插入元素value,若元素不在列表中或者列表不存在则不执行任何操作
\item LLEN key:获取列表key的长度,若key不存在则返回0,如果key不为列表类型则出错
\item LPOP key:用于移除并返回列表的第一个元素
\item LPUSH key value1 [value2]:将一个值(value1)或者多个值(value1,value2..)插入到列表key头部,若key不存在则创建一个新列表并执行操作
\item LPUSHX key value:将一个值(value)插入到列表key头部,若列表key不存在则操作无效
\item LRANGE key start stop:返回列表key中指定区间[start,stop]内的元素
\item LREM key count value:根据count的值,移除列表key中与value值相等的元素(即删除count个)
若$count > 0$从表头开始搜索,若$count < 0$从表尾开始搜索,若$count = 0$则移除key中所有与value相等的元素
\item LSET key index value:通过索引index来设置列表key中元素的值为value
\item LTRIM key start stop:对列表key进行修剪,只保留[start,stop]之间的元素,不在该区间的元素全部删除
\item RPOP key:移除并获取列表key中的最后一个元素
\item RPOPLPUSH source destination:移除并获取列表source中的最后一个元素,并添加到另一个列表destination中,返回该列表
\item RPUSH key value1 [value2]:将一个值(value1)或者多个值(value1,value2..)插入到列表key尾部,若key不存在则创建一个新列表并执行操作
\item RPUSHX key value:将一个值(value)插入到列表key尾部,若列表key不存在则操作无效
\end{itemize}

\par 而其中LRANGE,LLEN和LINDEX这三个命令与list的内容修改无关,因此在本文中不考虑这两个命令的相关OT操作。本文中只考虑剩余这11个命令的OT函数的设计和验证。
\subsection{Redis List API分类}
\par 经过分析,这11个命令可以根据操作类型和作用范围分为以下这三类。
\begin{itemize}
\item 单个元素的删除、修改、插入
\item 单个区间的删除、插入
\item 多个区间的删除
\end{itemize}
也就是在前一章中提到的三类操作,可以用这三类操作来表示这所有的11个命令,如下表所示:

\begin{table}[H]
\centering
\begin{tabular}{|c|c|c|} 
\hline
操作分类 &LIST命令 &具体操作\\
\hline
\multirow{7}*{第一类操作}  &$LPUSHX$  &$Ins(0,ele)$\\ 
\cline{2-3} 
&$RPUSHX$ & $Ins(len,ele)$\\
\cline{2-3}
&$LINSERT$ & $Ins(pos,ele)$\\
\cline{2-3}
&$LPOP$ & $Del(0)$\\
\cline{2-3}
&$RPOP$ & $Del(len-1)$\\
\cline{2-3}
&$RPOPLPUSH$ & $Del(len-1)$\\
\cline{2-3}
&$LSET$ & $Set(pos,ele)$\\
\hline

\multirow{2}*{第二类操作}  &$LPUSH$  &$Ins(0,str)$\\ 
\cline{2-3} 
&$RPUSH$ & $Ins(len,str)$\\
\hline

\multirow{2}*{第三类操作}  &$LTRIM$  &$Del(0,pos1-1;pos2+1,len-pos2-1)$\\ 
\cline{2-3} 
&$LREM$ & $Del(pos1,len1;pos2,len2;...;posk,lenk)$\\
\hline
\end{tabular}
\end{table}
注:$len$表示列表的长度
这样,由于Redis中list相关命令都可以用这三类操作来表示,所以我们只要设计完成这三类操作之间的所有OT函数,即可完成Redis中list相关命令的OT函数设计。
由于第一类函数和第二类函数已经设计完成,接下来只要完成剩余的OT函数设计即可。

\section{第三类 OT 函数设计}
\par 我们定义第三类函数为第三类的Del命令与第二类命令中的Ins操作之间的OT函数,以及第三类Del命令自身的OT函数,共2+1=3个。

\par 首先考虑第三类的Del命令与第二类命令中的Ins操作之间的OT函数,显然,我们需要按Ins操作的插入位置和Del操作的删除区间之间的关系进行分类,进行函数的设计。
\par Ins操作对于Del操作的转换,如果是插入位置位于删除区间中,则Ins操作转换为NOP,否则插入的位置要减去删除操作在该位置之前删除区间的总长度。
具体的数学公式如下:\\
$OT(Ins(p_{k+1},s_{k+1}),Del(p_1,l_1;p_2,l_2;...;p_k,l_k))$\\
\begin{equation}
= \begin{cases}
Ins(p_{k+1},s_{k+1}) \quad & p_{k+1} \le p_1 \\
no-op \quad & p_i<p_{k+1}<p_i+l_i\\
Ins(p_{k+1}-l_1-l_2-...-l_i,s_{k+1}) \quad & p_i+l_i \le p_{k+1} \le p_{i+1}\\
Ins(p_{k+1}-l_1-l_2-...-l_k,s_{k+1}) \quad & p_{k+1} \ge pk+lk \end{cases}\\
\end{equation}

\par Del操作对于Del操作的转换,如果是插入位置位于删除区间中,则该区间删除长度增加$|s|$,之前的区间不变,之后的区间都向后移动$|s|$单位长度
\par 如果插入位置不在删除区间中,那么在插入位置之前的区间不变,之后的区间都向后移动$|s|$单位长度
具体的数学公式如下:\\
$OT(Del(p_1,l_1;p_2,l_2;...;p_k,l_k),Ins(p_{k+1},s_{k+1}))$\\
\begin{equation}
= \begin{cases}
Del(p_1+|s_{k+1}|,l_1;p_2+|s_{k+1}|,l_2;...;p_i,l_i;p_{i+1}+|s_{k+1}|,l_{i+1};...;p_k+|s_{k+1}|,l_k)\\
  \quad \hfill p_{k+1} \le p_1 \\
Del(p_1,l_1;p_2,l_2;...;p_{i-1},l_{i-1};p_i,l_i+|s_{k+1}|;p_{i+1}+|s_{k+1}|,l_{i+1};...;p_k+|s_{k+1}|,l_k)\\
 \quad \hfill p_i<p_{k+1} < pi+li \\
Del(p_1,l_1;p_2,l_2;...;p_i,l_i;p_{i+1}+|s_{k+1}|,l_{i+1};...;p_k+|s_{k+1}|,l_k)\\
  \quad \hfill p_i+l_i \le p_{k+1} \le p_{i+1}\\
Del(p_1,l_1;p_2,l_2;...;p_k,l_k) \\
 \quad \hfill P_k+l_k \le p_{k+1} \\
 \end{cases}\\
\end{equation}

Del操作自身的转换时最为复杂的,一开始想要将要转换的Del操作中所有的区间一起转换,发现这样不仅做起来难度很大,而且写公式和用代码表达都很容易出错,但如果是将每个区间都用某个公式来转换,然后将转换后的新区间合并为新的删除操作,就可以方便的实现第三类Del操作自身的转换了。
显然,每个区间经转换后的新区间长度肯定是减少的(删除未被删除的内容),若该区间中所有内容都已经被删除则执行NOP。
需要考虑的问题是如果将每个区间单独进行转换,则区间之间会不会相互影响,由于原有区间都是互不重复的,即删除的内容互不相干,因此每个区间可以单独进行转换,转换完成后简单地将所有区间合并成新的Del操作即可。

转变思路后,第三类Del自身的转换就可以用第二类Del操作对于第三类Del操作的转换来代替了,减少了公式的复杂度,同时也增加了可读性。

与前面的设计类似,由删除区间与删除区间之间的位置关系进行函数的设计。
我们不妨假设目前待转换的删除区间端点为pos,区间长度为len,转换完成后的新区间端点为newpos,新区间长度为newlen,对于操作op2进行转换。
定性地进行分析,新区间即为待转换区间中没被op2删除的部分;定量地进行分析,newpos即为op2在pos之前的删除总长度,newlen为待转换区间中没有被op2删除的长度,这样新区间[newpos,newpos+newlen]即为未被op2删除的部分了。

下面两种表格中列出了经过转换后的新区间开始端点和新区间的长度,本文第4章中部分代码的编写便是参照这两张表格进行的。

\begin{table}[H]
\centering
\begin{tabular}{|c|c|c|} 
\hline
\multicolumn{3}{|c|}{$OT(Del(p_{k+1},l_{k+1}),Del(p_1,l_1;p_2,l_2;...;p_k,l_k))$}\\ 
\hline
$p_{k+1}$位置 &$p_{k+1}+l_{k+1}$位置 &转换后的$p_{k+1}$\\
\hline
$p_{k+1} < p_1$  &任意位置  &$p_{k+1}$\\ 
\hline
$p_i \le p_{k+1} < p_i+l_i$ &任意位置  &$p_i-l_1-l_2-...-l_{i-1}$\\ 
\hline
$p_i+l_i \le p_{k+1} < p_{i+1}$  &任意位置  &$p_{k+1}-l_1-l_2-...-l_{i-1}$\\ 
\hline
$p_{k+1} \ge p_k+l_k$  &任意位置  &$p_{k+1}-l_1-l_2...-l_k$\\ 
\hline
\end{tabular}
\end{table}


\begin{table}[H]
\centering
\begin{tabular}{|c|c|c|} 
\hline
\multicolumn{3}{|c|}{$OT(Del(p_{k+1},l_{k+1}),Del(p_1,l_1;p_2,l_2;...;p_k,l_k))$}\\ 
\hline
$p_{k+1}$位置 &$p_{k+1}+l_{k+1}$位置 &转换后的$l_{k+1}$\\
\hline
\multirow{4}*{$p_{k+1} < p_1$}  &$p_{k+1}+l_{k+1} \le p_1$  &$l_{k+1}$\\ 
\cline{2-3} 
&$p_j < p_{k+1}+l_{k+1} \le p_j+l_j$ &$p_j-l_1-l_2-...-l_{j-1}-p_{k+1}$\\
\cline{2-3}
&$p_j+l_j < p_{k+1}+l_{k+1} \le p_{j+1}$ & $l_{k+1}-l_1-l_2-...-l_j$\\
\cline{2-3}
&$p_{k+1}+l_{k+1} > P_k+l_k$ & $l_{k+1}-l_1-l_2-...-l_k$\\
\hline

\multirow{3}*{$p_i \le p_{k+1} < p_i+l_i$}  &$p_j < p_{k+1}+l_{k+1} \le p_j+l_j$  &$p_j-p_i-l_i-l_{i+1}...-l_{j-1}$\\ 
\cline{2-3} 
&$p_j+l_j < p_{k+1}+l_{k+1} \le p_{j+1}$ &$p_{k+1}+l_{k+1}-p_i-l_i-l_{i+1}-...-l_j$\\
\cline{2-3}
&$p_{k+1}+l_{k+1} > P_k+l_k$ & $p_{k+1}+l_{k+1}-p_i-l_i-l_{i+1}-...-l_k$\\
\hline

\multirow{3}*{$p_i+l_i \le p_{k+1} < p_{i+1}$}  &$p_j < p_{k+1}+l_{k+1} \le p_j+l_j$  &$p_j-p_{k+1}-l_{i+1}-l_{i+2}-...-l_{j-1}$\\ 
\cline{2-3} 
&$p_j+l_j < p_{k+1}+l_{k+1} \le p_{j+1}$ &$l_{k+1}-l_{i+1}-l_{i+2}-...-l_j$\\
\cline{2-3}
&$p_{k+1}+l_{k+1} > P_k+l_k$ & $l_{k+1}-l_{i+1}-l_{i+2}-...-l_k$\\
\hline
$p_{k+1} \ge p_k+l_k$  &任意位置  &$l_{k+1}$\\ 
\hline
\end{tabular}
\end{table}
从表格中不难发现,区间进行转换后新区间的开始端点与原区间的长度为关,而新区间的长度则与原区间的开始端点和长度都有关系.
具体的数学公式如下:

$OT(Del(p_{k+1},l_{k+1}),Del(p_1,l_1;p_2,l_2;...;p_k,l_k))$\\
\begin{equation}
= \begin{cases}
Del(p_{k+1},l_{k+1}) \quad &p_{k+1} < p_1 \quad \\ &p_{k+1}+l_{k+1} \le p_1 \\
Del(p_{k+1},p_j-l_1-l_2-...-l_{j-1}-p_{k+1}) \quad &p_{k+1} < p_1 \quad \\ &p_j < p_{k+1}+l_{k+1} \le p_j+l_j \\
Del(p_{k+1},l_{k+1}-l_1-l_2-...-l_j) \quad &p_{k+1} < p_1 \quad \\& p_j+l_j < p_{k+1}+l_{k+1} \le p_{j+1} \\
Del(p_{k+1},l_{k+1}-l_1-l_2-...-l_k) \quad &p_{k+1} < p_1 \quad \\& p_{k+1}+l_{k+1} > P_k+l_k  \\

Del(p_i-l_1-l_2-...-l_{i-1},p_j-p_i-l_i-l_{i+1}...-l_{j-1})        \quad &p_i \le p_{k+1} < p_i+l_i \quad \\ &p_j < p_{k+1}+l_{k+1} \le p_j+l_j \\
Del(p_i-l_1-l_2-...-l_{i-1},p_{k+1}+l_{k+1}-p_i-l_i-l_{i+1}-...-l_j) \quad &p_i \le p_{k+1} < p_i+l_i \quad \\& p_j+l_j < p_{k+1}+l_{k+1} \le p_{j+1} \\
Del(p_i-l_1-l_2-...-l_{i-1},p_{k+1}+l_{k+1}-p_i-l_i-l_{i+1}-...-l_k) \quad &p_i \le p_{k+1} < p_i+l_i \quad \\& p_{k+1}+l_{k+1} > P_k+l_k  \\


Del(p_{k+1}-l_1-l_2-...-l_{i-1},p_j-p_{k+1}-l_{i+1}-l_{i+2}-...-l_{j-1})    \quad &p_i+l_i \le p_{k+1} < p_{i+1} \quad \\& p_j < p_{k+1}+l_{k+1} \le p_j+l_j \\
Del(p_{k+1}-l_1-l_2-...-l_{i-1},l_{k+1}-l_{i+1}-l_{i+2}-...-l_j)    \quad &p_i+l_i \le p_{k+1} < p_{i+1} \quad \\& p_j+l_j < p_{k+1}+l_{k+1} \le p_{j+1} \\
Del(p_{k+1}-l_1-l_2-...-l_{i-1},l_{k+1}-l_{i+1}-l_{i+2}-...-l_k)    \quad &p_i+l_i \le p_{k+1} < p_{i+1} \quad \\& p_{k+1}+l_{k+1} > P_k+l_k  \\
Del(p_{k+1}-l_1-l_2...-l_k,l_{k+1}) \quad &p_{k+1} \ge p_k+l_k
\\&(i \ge j)
 \end{cases}\\
\end{equation}



\section{其它 OT 函数设计}
\par 显然,第三类的del命令可以涵盖第一类和第二类的del命令,第二类的ins命令可以涵盖第一类的ins命令,因此我们最后只要考虑set(pos,ele)命令和第二类的ins操作、第三类的del操作之间的OT函数关系,即可覆盖完全所有的OT函数设计。共有2+2=4个函数。

\par Ins操作对于Set操作的转换就是其本身。
\begin{equation}
OT(Ins(i,s),Set(j,x))= Ins(i,s)\\
\end{equation}

\par Set操作对于Ins操作的转换,如果插入位置在Set位置之前则Set位置要增加$|s|$单位长度,否则Set操作不变。
\begin{equation}
OT(Set(i,x),Ins(j,s))= \begin{cases}
{Set(i,x)}  \quad & i<j \\
{Set(i+|s|,x)} \quad  &i \ge j \end{cases} \\
\end{equation}

\par Del操作对于Set操作的转换就是其本身。
\begin{equation}
OT(Del(p_1,l_1;p_2,l_2;...;p_k,l_k),Set(p_{k+1},x)) = Del(p_1,l_1;p_2,l_2;...;p_k,l_k)\\
\end{equation}

\par Set操作对于Del操作的转换,如果是Set位置位于删除区间中,则Ins操作转换为NOP,否则插入的位置要减去删除操作在该位置之前删除区间的总长度。
$OT(Set(p_{k+1},x),Del(p_1,l_1;p_2,l_2;...;p_k,l_k))$
\begin{equation}
= \begin{cases}
Ins(p_{k+1},s_{k+1}) \quad & p_{k+1} < p_1 \\
no-op \quad & p_i \le p_{k+1} < p_i+l_i\\
Set(p_{k+1}-l_1-l_2-...-l_i,x) \quad & p_i+l_i \le p_{k+1} < p_{i+1}\\
Set(p_{k+1}-l_1-l_2-...-l_k,x) \quad & p_{k+1} \ge pk+lk \end{cases}\\
\end{equation}