\chapter{结论与未来工作}
\par Redis List命令的OT函数设计经过验证是成功的,其满足CP1正确性。
\section{工作总结}
本文所做实验论证了Redis系统中List相关命令的OT函数的设计与验证
\section{研究展望}
验证所设计的OT函数正确性只是一个起点,要真正能够在Redis系统中实现List的OT函数,还有许多工作亟待完成。
\subsection{OT函数的实现}
由于Reids系统的主从结构和我们的要求不符合,要想在Redis中实现OT函数,还需要修改Redis的系统通信框架,使其变为P2P的通信结构。
\subsection{验证代码的改进}
此外,由于在本次实验的TLA+代码中大量使用了递归的函数或者定义,使得对于第三类函数以及全部函数正确性验证的过程时间较长,效率较低,代码还存在着改进的空间。
之所以采用TLA+来完成OT函数的验证,正是因为其可以遍历所有操作的特性,改进验证代码目前有两个想法:一是使用语法糖Pluascal从而避免对于递归定义的依赖,二是使用其他编程语言来编写验证代码。