\chapter{结论与未来工作}
\section{工作总结}
\par 本文首先基于协同编辑的应用背景,以及Replicated List规约,引出了OT的定义和OT系统的结构,以及Redis系统中list相关命令的应用需求。接着,本文回顾了相关工作,分析了他们的不足。针对这些不足,本文研究了Redis系统中List相关命令,实现了其OT函数的设计与验证,主要贡献包括:
\par 首先将Redis List的相关命令进行基于坐标和操作方式的分类,分成三大类操作,能够更方便地实现OT函数的设计。
\par 其次对于所有简化后的命令进行了OT函数的数学公式设计,补全了目前的研究中所欠缺的部分。
\par 最后使用一种新的证明方式,即通过TLA+实现了对于所设计OT函数的证明,并且对实验结果进行相应的分析。对于第一类函数和第二类函数的验证都很顺利,对于第三类函数,由于在代码中大量使用了递归的结构,导致运行的效率不是很高,验证成功的列表长度有限,不过已经可以证明数学公式的正确性了。
\section{研究展望}
验证所设计的OT函数正确性只是一个起点,要真正能够在Redis系统中实现List的OT函数,还有许多工作亟待完成。
\subsection{OT函数的实现}
由于Reids系统的主从结构和我们的要求不符合,要想在Redis中实现OT函数,还需要修改Redis的系统通信框架,使其变为P2P的通信结构。
\subsection{验证代码的改进}
此外,由于在本次实验的TLA+代码中大量使用了递归的函数或者定义,使得对于第三类函数以及全部函数正确性验证的过程时间较长,效率较低,代码还存在着改进的空间。
之所以采用TLA+来完成OT函数的验证,正是因为其可以遍历所有操作的特性,改进验证代码目前有两个想法:一是使用语法糖Pluascal从而避免对于递归定义的依赖,二是使用其他编程语言来编写验证代码。