\chapter{相关工作}
\section{OT 函数的性质}
	OT 函数需要满足的性质:
	\begin{itemize}
	  \item Convergence Property 1(CP1): 这是Jupiter 协议正确性的必要条件。对于定义在文档状态S上的给定操作O1和O2,满足CP1等式:$S \circ O1 \circ OT(O2,O1) = S \circ O2 \circ OT(O1,O)$,也就是说在S上按顺序实施O1,OT(O2,O1)操作和实施O2,OT(O1,O2)操作效果相同。
	  \item Convergence Property 2(CP2):对于定义在文档状态S上的给定操作O1和O2,满足CP2等式:$OT(OT(O1,O2),OT(O3,O2)) = OT(OT(O1,O3),OT(O2,O3))$
	  \item Inverse Property 1(IP1)
	  \item Inverse Property 2(IP1)
	  \item Inverse Property 3(IP3)
	\end{itemize}
	
\section{OT函数的设计}
可以设计OT函数,并通过控制算法的调用来保证最终结果的一致性,现在ins,del,set等简单operation的OT函数已经基本实现,ins,del单个区间的OT函数也已经基本上设计完成。
\subsection{第一类 OT 函数的设计}
我们定义第一类函数为第一类命令之间的OT函数,即Ins,Del,Set三种操作之间的OT函数,共有3*3=9个。
\begin{equation}
\begin{aligned}
Set \begin{cases}
OT(set (i,x), set (j,y)) =\begin{cases}
    no-op \quad &pr1 > pr2 \quad i=j\\
	{set (i,x)} \quad &else \end{cases} \\ 
OT(Set(i,x),Ins(j,y))=\begin{cases}
{Set(i,x)}  \quad &i<j\\
{Set(i+1,x)} \quad  &i\ge j \end{cases} \\
OT(Set(i,x),Del(j))=\begin{cases}
{Set(i,x)} \quad &i<j\\
{no-op} \quad & i=j\\
{Set(i-1,x)} \quad &i>j \end{cases} \\
\end{cases}
\end{aligned}
\end{equation}
\begin{equation}
\begin{aligned}
Ins \begin{cases}
OT(Ins(i,x), set (j,y)) =
{Ins(i,x)}\\
OT(ins (i,x), ins (j,y)) =\begin{cases}
	{ins(i+1, x)}   \quad & i > j\\
	{ins(i, x)}    \quad & i < j\\
	{ins(i+1, x)}   \quad  & i = j \quad pr1 < pr2\\
	{ins(i, x)}   \quad  & i = j \quad pr1 > pr2 \end{cases} \\
OT(Ins(i,x),Del(j))=\begin{cases}
{Ins(i,x)}  \quad i \le j\\
{Ins(i-1,x)} \quad i>j \end{cases}\\
\end{cases}
\end{aligned}
\end{equation}


\begin{equation}
\begin{aligned}
Del \begin{cases}
OT(Del (i), Set (j,x)) =
	{Del(i)}\\
OT(Del (i), Ins (j,x)) =\begin{cases}
	{Del (i+1)}  \quad &i \ge j\\
	{Del (i)}   \quad &i < j\\ \end{cases}\\
OT(del (i), del (j)) =\begin{cases}
	{Del (i-1)} \quad &i > j\\
	{Del (i)} \quad &i < j\\
	{no-op}   \quad &i = j \end{cases}\\
\end{cases}
\end{aligned}
\end{equation}

\subsection{第二类 OT 函数设计}
我们定义第二类函数为第二类命令之间的OT函数,即Ins,Del两种命令之间的OT函数,共2*2=4个。
\begin{equation}
OT(Ins(p1,s1),Ins(p1,s2))= \begin{cases}
Ins(p1,s1) \quad & p1<p2 \\
Ins(p1+ |s2|,s1) \quad & p1>p2 \\
Ins(p1+ |s2|,s1) \quad & p1=p2 \quad pr1<pr2 \\
Ins(p1,s1) \quad & p1=p2 \quad pr1>pr2
 \end{cases}\\
\end{equation}

\begin{equation}
OT(Ins(p1,s1),Del(p2,l1))= \begin{cases}
Ins(p1,s1) \quad & p1 \le p2 \\
no-op \quad & p2<p1<p2+l1\\
Ins(p1-l1,s1) \quad & p1 \ge p2+l1 \end{cases}\\
\end{equation}

\begin{equation}
OT(Del(p1,l1),Ins(p2,s1))= \begin{cases}
Del(p1,l1) \quad & p1 + l1 \le p2 \\
Del(p1,l1+|s1|) \quad & p1<p2<p1+l1 \\
Ins(p1+ |s1|,l1) \quad & p1 \ge p2 \end{cases}\\
\end{equation}

\begin{equation}
OT(Del(p1,l1),Del(p2,l2))= \begin{cases}
Del(p1,l1) \quad & p1<p2 \quad p1+l1 \le p2 \\
Del(p1,p2-p1) \quad & p1<p2 \quad p2<p1+l1 \le p2+l2\\
Del(p1,l1-l2) \quad & p1<p2 \quad p2+l2<p1+l1\\
no-op \quad & p2 \le p1 < p2+l2 \quad p1+l1 \le p2+l2\\
Del(p2,p1+l1-p2-l2) \quad & p2 \le p1 <p2+l2 \quad  p1+l1>p2+l2\\
Del(p1-l2,l1) \quad & p1 \ge p2+l2  \end{cases}\\
\end{equation}
第二类简单例子

\section{OT 函数的验证}

	