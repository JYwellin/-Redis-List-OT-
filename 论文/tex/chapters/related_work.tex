\chapter{相关工作}
\label{chapter:related_work}
\section{OT 函数的性质}
	过去的OT研究已经建立了一套完整的OT算法正确性的性质标准~\cite{Sun:CSCW14},
	违反任何转换性质会导致错误的转换,从而可能使得文档不一致。我们关注的性质:
	\begin{itemize}
	  \item Convergence Property 1(CP1): 这是Jupiter 协议正确性的必要条件。对于定义在文档状态S上的给定操作O1和O2,满足CP1等式:$S \circ O1 \circ OT(O2,O1) = S \circ O2 \circ OT(O1,O)$,也就是说在S上按顺序实施O1,OT(O2,O1)操作和实施O2,OT(O1,O2)操作效果相同。
	  \item Convergence Property 2(CP2):对于定义在文档状态S上的给定操作O1和O2,满足CP2等式:$OT(OT(O1,O2),OT(O3,O2)) = OT(OT(O1,O3),OT(O2,O3))$
	  \item Inverse Property 1(IP1)
	  \item Inverse Property 2(IP1)
	  \item Inverse Property 3(IP3)
	\end{itemize}
	关注这五个特性的原因是CP1,CP2是收敛性的变换性质,IP1、IP2和IP3是与撤销相关的变换性质。CP1和CP2被发现是在OT系统中实现收敛性的充分必要条件,允许操作以任何顺序进行转换,不过并不是所有OT系统都需要满足这两个条件。CP1在某些OT系统中不是必须的,例如GOT~\cite{Sun:CSCW98},这些OT系统从不以不同的顺序变换两个操作;CP2同样也不是必须的,比如Jupiter和Google Docs等OT系统,这些OT系统会在操作中施加特殊的转换命令。由于是基于Jupiter协议所作的OT函数设计,所以本文中只验证所设计函数的CP1正确性。
	
\section{OT函数的设计}
可以设计OT函数,并通过控制算法的调用来保证最终结果的一致性。本文中将Ins(插入),Del(删除),Set(设置)单个元素的简单操作称为第一类操作,将Ins,Del单个区间的操作称为第二类操作,作用于多个区间的操作称为第三类操作(本文中只讨论删除多个区间的操作)。
假设list当前状态为s,那么具体操作的定义如下:
\begin{table}[H]
\centering
\begin{tabular}{|c|c|c|} 
\hline
操作分类 &操作类型 &操作效果\\
\hline
\multirow{3}*{第一类操作}  &$Del(i)$  &删除第i个元素\\ 
\cline{2-3} 
&$Ins(i,x)$ &在位置i插入元素x\\
\cline{2-3}
&$Set(i,x)$ &将位置i上的元素设为x\\
\hline

\multirow{2}*{第二类操作}  &$Ins(p,s)$  &在位置i插入串s\\ 
\cline{2-3} 
&$Del(p,l)$ &从位置p开始删除长度为l的串\\
\hline
第三类操作 &$Del(p_1,l_1;p_2,l_2;...;p_k,l_k)$ &删除区间$[p_1,p_1+l_1],...,[p_k,p_k+l_k]$内的元素\\

\hline
\end{tabular}
\end{table}

现在Ins,Del,Set等简单操作相互之间的OT函数已经设计好,Ins,Del单个区间的OT函数也已经基本上设计完成。
\subsection{第一类 OT 函数的设计}
我们定义第一类函数为第一类操作之间的OT函数,即Ins,Del,Set三种操作之间的OT函数,共有3*3=9个。
这里列出了Del操作对应于Ins,Del,Set的OT函数,全部9个函数的数学公式详见附录。\cite{Sun:CSCW14}
\begin{equation}
\begin{aligned}
Del \begin{cases}
OT(Del (i), Set (j,x)) =
	{Del(i)}\\
OT(Del (i), Ins (j,x)) =\begin{cases}
	{Del (i+1)}  \quad &i \ge j\\
	{Del (i)}   \quad &i < j\\ \end{cases}\\
OT(del (i), del (j)) =\begin{cases}
	{Del (i-1)} \quad &i > j\\
	{Del (i)} \quad &i < j\\
	{no-op}   \quad &i = j \end{cases}\\
\end{cases}
\end{aligned}
\end{equation}
套用公式解释前言中举例的OT变换,$OT(del (4), del (2)) = del(3)$,$OT(del (2), del (4)) = del(2)$,与例子相符。

\subsection{第二类 OT 函数设计}
我们定义第二类函数为第二类操作之间的OT函数,即Ins,Del两种区间操作之间的OT函数,共2*2=4个。
这里列出了Ins操作对应于Ins,Del的OT函数,全部4个函数的数学公式详见附录。\cite{Sun:CSCW14}
\begin{equation}
OT(Ins(p1,s1),Ins(p2,s2))= \begin{cases}
Ins(p1,s1) \quad & p1<p2 \\
Ins(p1+ |s2|,s1) \quad & p1>p2 \\
Ins(p1+ |s2|,s1) \quad & p1=p2 \quad pr1<pr2 \\
Ins(p1,s1) \quad & p1=p2 \quad pr1>pr2
 \end{cases}\\
\end{equation}
Ins自身的
\begin{equation}
OT(Ins(p1,s1),Del(p2,l1))= \begin{cases}
Ins(p1,s1) \quad & p1 \le p2 \\
no-op \quad & p2<p1<p2+l1\\
Ins(p1-l1,s1) \quad & p1 \ge p2+l1 \end{cases}\\
\end{equation}

\section{OT 函数的验证}

CP1与CP2性质要求OT函数的设计者考虑所有可能的操作组合情况,
并且,对一种特定的操作组合,还要考虑所有可能的访问位置的组合情况。
OT函数的设计易于出错。
因此,有一系列工作致力于如何验证OT函数的正确性。
Imine 等人~\cite{Imine:TCS06} 基于代数规约框架开发了OT函数自动验证工具 VOTE。
该工具可以自动验证给定的OT函数的正确性或者给出反例。
Imine 等人使用该工具检测了 GROVE~\cite{Ellis:SIGMOD89}, 
Joint Emacs~\cite{Ressel:CSCW96}, REDUCE~\cite{Sun:TOCHI98}, 
CoWord, SAMS~\cite{Molli:CSCW02} 和 So6~\cite{Sun:CSCW04} 协同编辑器。
结果发现,这六个协同编辑器所使用的OT函数都不满足CP2性质,
而且 GROVE 和 Joint Emacs 的OT函数也不满足CP1性质。
Sun 等人~\cite{Sun:CSCW14} 开发了一套验证框架和一个验证工具,
能够穷举所有可能的操作转换情况。
基于此,作者不仅重现了之前已有工作中指出的OT函数设计的错误之处,
而且揭示了所有可能的OT函数的设计错误。
在其后续工作中,Sun 等人~\cite{Sun:CSCW17} 推广了该方法,
用于验证支持字符串插入与字符串删除的列表OT函数的正确性。
在这两份工作中,作者不仅考虑了CP1和CP2性质,
还考虑了OT函数可能需要满足的其它性质。
Randolph 等人~\cite{Randolph:arXiv13} 从理论上考察满足CP1和CP2性质的OT函数的存在性问题。
作者将该问题转化为 (Controller Synthesis) 合成问题,
然后尝试借助形式化验证工具 Uppaal-Tiga~\cite{Cassez:CONCUR05} 自动合成OT函数。
结果表明,在不使用位置(position)之外的信息的情况下,不存在满足CP2性质的OT函数。
Liu 等人~\cite{Liu:FM14} 利用模型检验技术验证了某些OT函数满足CP1性质,
也重现了违反CP2性质的OT函数的设计错误。
此外,作者还第一次验证了一类基于操作上下文 (Operation Context) 概念~\cite{Sun:CSCW06} 的OT控制算法的正确性。