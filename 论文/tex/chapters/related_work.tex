\chapter{相关工作}

\section{OT-based 协议}
client-server 结构


一个OT系统包含2个关键的部分,上层的控制算法和底层的OT函数。

控制算法负责决定哪些操作以何种顺序进行转换,而具体的OT函数则实施具体的两个操作之间的变化。
OT函数的数量由OT系统的模型所支持的数据和操作类型所决定。
这两者由一系列转换条件和属性结合在一起,所以整个OT系统的正确性就是由控制算法和OT函数的正确性以及协议的正确性所共同决定的。

由于这种分层结构,我们可以单独考虑OT函数的设计,而不必关心控制算法。

\subsection{OT 函数的性质}
	OT 函数需要满足的性质:
	\begin{itemize}
	  \item Convergence Property 1(CP1): Jupiter 协议正确性的必要条件。
	  \item Convergence Property 2(CP2):
	  \item Inverse Property 1(IP1): (IP1,2,3 只提及,不需给定义)
	  \item Inverse Property 2(IP1):
	  \item Inverse Property 3(IP3):
	\end{itemize}

\section{OT函数的设计}
可以设计OT函数,并通过控制算法的调用来保证最终结果的一致性,现在ins,del,set等简单operation的OT函数已经基本实现,ins,del单个区间的OT函数也已经基本上设计完成,详见第三章中的第一类和第二类OT函数设计。
	
第一类简单例子

第二类简单例子

(第一类、第二类完整例子见附录)
\section{OT 函数的验证}

	