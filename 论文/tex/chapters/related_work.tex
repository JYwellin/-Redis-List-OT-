\chapter{相关工作}
\section{系统模型}
	\par 一个OT系统包含2个关键的部分,上层的控制算法和底层的OT函数。控制算法负责决定哪些操作以何种顺序进行转换,而具体的OT函数则实施具体的两个操作之间的变化。OT函数的数量由OT系统的模型所支持的数据和操作类型所决定。这两者由一系列转换条件和属性结合在一起,所以整个OT系统的正确性就是由控制算法和OT函数的正确性以及协议的正确性所共同决定的。
\subsection{转换协议}
	\par 已有的OT研究建立了一系列协议作为OT算法正确性的标准,违反任意一种协议都会导致不正确的转换结果,从而使得文档的编辑不能实现一致。有如下协议:
	\begin{itemize}
	\item Convergence Property 1(CP1):对于定义在文档状态S上的给定操作O1和O2,满足CP1等式:$S \circ O1 \circ OT(O2,O1) = S \circ O2 \circ OT(O1,O)$,也就是说在S上按顺序实施O1,OT(O2,O1)操作和实施O2,OT(O1,O2)操作效果相同。
	\item Convergence Property 2(CP2):对于定义在文档状态S上的给定操作O1和O2,满足CP2等式:$OT(OT(O1,O2),OT(O3,O2)) = OT(OT(O1,O3),OT(O2,O3))$
	\item Inverse Property 1(IP1):对于定义在文档状态S上的操作,!O为操作O的相反操作,满足IP1等式:$S \circ O \circ !O = S$,也就是说在S上执行O,!O操作后文档状态不发生改变。
	\item Inverse Property 2(IP2):对于定义在文档状态S上的给定操作O1和O2,!O2为操作O2的相反操作,满足IP2等式:$OT(OT(O1,O2),!O2) = O1$,也就是说O1按顺序对照O2,!O2进行转换后仍未O1本身。
	\item Inverse Property 3(IP3):对于定义在文档状态S上的给定操作O1和O2,!O为操作O的相反操作,满足IP3等式:$OT(!O2,O1)=!(O2,O1)$
	\end{itemize}
	
\section{已有OT函数的设计}
	\par 可以设计OT函数,并通过控制算法的调用来保证最终结果的一致性,现在ins,del,set等简单operation的OT函数已经基本实现,ins,del单个区间的OT函数也已经基本上设计完成,详见第三章中的第一类和第二类OT函数设计。
	
	