\chapter{前言}
本文基于协同编辑的应用背景,出于实际需求,参考 Redis 开源系统中的列表数据结构和Jupiter协议,完成了OT函数的设计和验证。

\textcolor{red}{可再详细一点,重点在于为什么做。可以将摘要缩写一下。注意:像OT这种缩写词,第一次出现之前要有介绍。}

\section{应用背景:协同编辑应用}

协同编辑系统,如 Google Docs~\footnote{https://docs.google.com}、Apache Wave~\footnote{https://incubator.apache.org/wave/}、wikis~\cite{Leuf:Wiki01} 
等允许不同地点的用户同时编辑同一份文档。
为了获得较快的响应和较高的可用性,系统会在不同的地点或设备进行文档的复制。
用户可以在某个副本上进行文档的编辑,并将做出的修改异步地传递给其他副本。
本地操作可以立即执行, 不必等待服务器处理完再响应用户操作。
系统必须保证编辑的一致性,即在所有用户完成文档的编辑后,所有的副本内容一致。

\section{技术背景}

复制列表 (Replicated List) 经常被用以建模协同编辑应用的核心功能,
如插入字符、删除字符等~\cite{Ellis:SIGMOD89, Nichols:UIST95, Attiya:PODC16}。
收敛性 (Convergence) 是复制列表的一种常用规约~\cite{Ellis:SIGMOD89}。
基于操作转换 (Operational Transformation; 简称 OT) 的思想,
研究者提出了多种实现收敛性的协议~\cite{Ellis:SIGMOD89, Prakash:TOCHI94, Nichols:UIST95, Sun:TOCHI98, Sun:CSCW98, 
Vidot:CSCW00, Shen:CSCW02, Sun:TOCHI02, Li:ICPADS04, Sun:TPDS09, Sun:CSCW14}。

\subsection{复制列表的规约}
复制列表至少需要支持三类常见操作: 
(1) Insert: 在某指定位置插入给定字符;
(2) Delete: 删除某指定位置上的字符;
(3) Read: 读取整个列表。
%
Ellis 等人在1989年提出了复制列表的收敛性规约~\cite{Ellis:SIGMOD89}。
收敛性规约是对``系统静止状态''~\cite{Ellis:SIGMOD89}的一种约束。
\begin{definition}
  对于一个复制系统而言,如果所有操作都已在各个副本节点上得以处理
  ——换句话说,没有操作仍在传输之中或者等待处理——则称系统处于静止状态。
\end{definition}

\begin{definition}
  如果一个复制系统处于静止状态时,
  所有副本节点的状态(即列表内容)相同,
  则称该系统满足收敛性。
\end{definition}

在收敛性的基础上,研究人员提出了另一种稍强的复制列表规约,
称为强最终一致性 (Strong Eventual Consistency)~\cite{Shapiro:SSS11}。
\begin{definition}
  如果两个副本处理了相同的操作集合,那么它们的状态(即列表内容)必须相同。
\end{definition}

由于大多数的复制列表协议都满足强最终一致性,
所以下文所提到的收敛性指的都是强最终一致性。

\subsection{操作转换思想}
操作转换Operational Transformation (OT)是一种为了支持协作功能,在协作软件系统中所采用的技术。
OT最早在1989年被提出,是为了在纯文本文档的协同编辑中实现一致性和并发控制所发明的。
经过二十余年的研究,OT的能力已经得到了拓展,在2009年OT做为一种核心技术被Apache Wave和Google Docs所采用来实现协同编辑。
OT的基本思想是根据先前执行的并发操作的影响,对正在编辑的操作的参数进行转换或调整,使转换后的操作能够保持文档的一致性。
考虑如下例子:

\begin{figure}[t]
\begin{minipage}[t]{0.5\linewidth}
\centering
\includegraphics[width=2.6in]{figures/exm1.bmp}
\caption{不使用OT}
\label{fig:side:a}
\end{minipage}%
\begin{minipage}[t]{0.5\linewidth}
\centering
\includegraphics[width=2.6in]{figures/exm2.bmp}
\caption{使用OT}
\label{fig:side:b}
\end{minipage}
\end{figure}

如图所示,在某个状态S时,list内容为”ABCDE"。
此时client端执行了一个删除第4个字符的操作,server端执行了一个删除第2个字符的操作,如图所示,client端list的内容变为“ABCE”,server端list的内容变为“ACDE”。
如果不进行转换,直接将操作相互传递,则client端接受到的操作是删除第2个字符,server端接受到的操作是删除第4个字符,完成操作后client端list的内容为“ACE”,server端list的内容为“ACD”,此时list的内容不同,不满足一致性。其原因在于client端的删除第4个字符的操作应该是删除D,但是传递到server端时,由于没有进行操作的转换,此时删除第4个字符即删除E,造成了删除内容的不一致。

使用OT技术将$Del\ 4$转换为$Del\ 3$,$Del\ 2$转换为$Del\ 2$,这样操作的传递完成后,文档的内容仍然是一致的。

\subsection{基于``操作转换''的协议}
一个OT系统通常包含2个关键的部分: 上层的控制算法和底层的OT函数。

\begin{figure}[H]
\centering
\includegraphics{figures/structure.bmp}
\caption{OT系统结构}
\end{figure}

控制算法负责决定哪些操作以何种顺序进行转换,而具体的OT函数则实施具体的两个操作之间的变化。
OT函数的数量由OT系统的模型所支持的数据和操作类型所决定。
这两者由一系列转换条件和属性结合在一起,所以整个OT系统的正确性就是由控制算法和OT函数的正确性以及协议的正确性所共同决定的。

由于这种分层结构,我们可以单独考虑OT函数的设计,而不必关心控制算法。

本文中OT函数的设计与验证基于Jupiter协议~\cite{Nichols:UIST95}。

Jupiter是一个多用户协同软件,支持长期的远程协作,支持共享文档、共享工具以及在线的音频视频通信。

Jupiter采用client-server结构,即中央服务器存储虚拟对象的状态和执行相关的程序代码,客户端进行输入输出等工作,Jupiter算法也是基于主从结构来设计和实现的。并且Jupiter采用积极的并发控制算法,即不必要等待服务器处理完再响应用户操作,本地操作可以立即执行,然后再将执行的结构异步地进行传递,如果同时有两个或以上的参与者进行修改,就需要采取冲突解决算法来进行相应的操作调整。与其他gropuware system不同,Jupiter不使用直接建立在客户端之间的同步协议,而是使每个客户端与服务器同步,服务器将某个客户端所做的修改经过转换分享传递给其他用户。

Jupiter算法依赖于OT来修复冲突的消息,保证一致性的实现。可以用状态图来表示server和client的状态,若client处于(x,y),表示此时client端已经产生并执行x两条自己的操作,并且从server接受并执行了y条操作,server端会为每个client端维护一个这样的二维空间,如果执行顺序一致,则server和client在状态图上行进的路径是一直的,所在位置也相同,但如果发生冲突,行进路径就会发生分歧,此时可以接收来自对方的经过OT转换的操作,使两者达到相同的状态,并且算法可以保证,不管server端和clinet端在状态图上的分歧有多远,只要接收来自对方的经过OT转换的操作,达到相同的状态,那么两者数据的值就一定是相同的,满足一致性的要求。

\section{Redis 系统}
\subsection{Redis简介}
Redis是一个开源的、可基于内存也可持久化的数据结构存储系统,可以用作数据库、缓存和消息中间件,并且提供多种语言的API。Redis支持多种类型的数据结构,如字符串(string),散列(hash),列表(即本文中所关心的lists),集合(set),有序集合(sorted set)等。
\subsection{Redis特点}
与其他key-value缓存产品相比,Redis具有以下的几个特点:
\begin{itemize}
\item Redis支持数据的持久化,可以在磁盘中保存内存里的数据
\item 除了简单的key-value型数据,Redis还提供多种数据结构的存储(包括List)
\item Redis支持主从模式的数据备份
\item 可以对Redis中的各种数据结构执行原子操作
\end{itemize}
\section{本文研究工作: 面向 Redis List的OT函数的设计与验证}
	\par 本次毕业设计的目标是实现Redis List所支持的14种非阻塞操作的OT(Operational Transformation)函数,并且对实现函数的正确性进行验证。阿里云和RedisLab的团队目前都在对Redis List的操作进行开发,Redis List操作的OT函数实现具有应用前景和商业前途。

	\par 在OT函数的设计方面,本文使用数学公式表示出所有OT函数的基本形式,并以图片和表格进行说明。

	\par 在OT函数的验证方面,使用TLA +完成了对所设计OT函数的验证,证明其满足CP1正确性,并且对验证代码的复杂度进行了相应分析。
\section{论文组织}
	\par 本文后续内容组织如下:
	\par 第\ref{chapter:related_work}章介绍本文的相关工作,包括系统模型和已有的相关OT函数的设计。
	\par 第\ref{chapter:design}章介绍了Redis列表相关的基本命令,并对其进行了分类,然后进行了对应OT函数的设计。
	\par 第\ref{chapter:proof}章介绍了TLA+,并使用TLA+完成了对上述设计好的OT函数的验证,对实验结果进行了分析。
	\par 第\ref{chapter:conclusion_future_work}章是本论文的结论和以后工作的相关展望。