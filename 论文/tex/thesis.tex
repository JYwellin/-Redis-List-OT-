\documentclass[bachelor,oneside,winfonts]{njuthesis}
\usepackage{epsfig}
\usepackage[linesnumbered,ruled,vlined]{algorithm2e}
\usepackage{amsmath}
\usepackage[noend]{algpseudocode}
\usepackage{inparalist}
\usepackage{tikz}
\SetKwBlock{Begin}{Upon}{end}
\makeatletter
\def\BState{\State\hskip-\ALG@thistlm}
\makeatother
\usepackage{diagbox}


%%%%%%%%%%%%%%%%%%%%%%%%%%%%%%%%%%%%%%%%%%%%%%%%%%%%%%%%%%%%%
% 设置论文的中文封面

% 论文标题,不可换行
\title{面向Redis list的OT函数的设计与验证}
% 如果论文标题过长,可以分两行,第一行用\titlea{}定义,第二行用\titleb{}定义,将上面的\title{}注释掉
%\titlea{分布式移动计算中数据共享技术研究}
%\titleb{}

% 论文作者姓名
\author{纪业}
% 论文作者联系电话
\telphone{18362917059}
% 论文作者电子邮件地址
\email{451686157@qq.com}
% 论文作者学生证号
\studentnum{141220044}
% 论文作者入学年份(年级)
\grade{2014}
% 导师姓名职称
\supervisor{魏恒峰~讲师}
% 导师的联系电话
\supervisortelphone{}
% 论文作者的学科与专业方向
\major{计算机应用}
% 论文作者的研究方向
\researchfield{分布式算法}
% 论文作者所在院系的中文名称
\department{计算机科学与技术系}
% 论文作者所在学校或机构的名称。此属性可选,默认值为``南京大学''。
\institute{南京大学}
% 论文的提交日期,需设置年、月、日。
\submitdate{2018年6月20日}
% 论文的答辩日期,需设置年、月、日。
\defenddate{2018年6月7日}
% 论文的定稿日期,需设置年、月、日。此属性可选,默认值为最后一次编译时的日期,精确到日。
%% \date{2013年5月1日}

%%%%%%%%%%%%%%%%%%%%%%%%%%%%%%%%%%%%%%%%%%%%%%%%%%%%%%%%
% 设置论文的英文封面-

% 论文的英文标题,不可换行
\englishtitle{Design and Verification of OT Function for Redis List}
% 论文作者姓名的拼音
\englishauthor{Ji Ye}
% 导师姓名职称的英文
\englishsupervisor{Reasearch Assistant Wei Henfeng}

% 论文作者学科与专业的英文名
\englishmajor{Computer Application}
% 论文作者所在院系的英文名称
\englishdepartment{Department of Computer Science and Technology}
% 论文作者所在学校或机构的英文名称。此属性可选,默认值为``Nanjing University''。
\englishinstitute{Nanjing University}
% 论文完成日期的英文形式,它将出现在英文封面下方。需设置年、月、日。日期格式使用美国的日期
% 格式,即``Month day, year'',其中``Month''为月份的英文名全称,首字母大写;``day''为
% 该月中日期的阿拉伯数字表示;``year''为年份的四位阿拉伯数字表示。此属性可选,默认值为最后
% 一次编译时的日期。
\englishdate{\today}


\begin{document}
	
	% 制作中文封面
	\maketitle
	% 制作英文封面
	\makeenglishtitle
	
	%%%%%%%%%%%
	% 开始前言部分
	\frontmatter
	
	%摘要
	%%%%%%%%%%%%%%%%%%%%%%%%%%%%%%%%%%%%%%%%%%%%%%%%%%%%%%%%%%%%%%%%%%%%%%%%%%%%%%%
% 论文的中文摘要
\begin{abstract}
  协同编辑系统允许处于不同地点的多用户同时编辑同一份文档。
  为了获得较快的响应和较高的可用性,该类系统通常采用副本复制技术,
  即将文档副本存放在不同的节点上。
  用户可以在某个本地(或就近的)副本节点上编辑文档,并将做出的修改异步地传递给其他副本节点。
  与此同时,协同编辑系统必须保证文档的一致性。

  复制列表 (Replicated List) 经常被用以建模协同编辑应用的核心功能,如插入字符、删除字符等。
  收敛性 (Convergence) 是复制列表的一种常用规约,是对文档一致性要求的形式化描述。
  基于操作转换 (Operational Transformation; 简称 OT) 的思想, 研究者提出了多种实现收敛性的协议。
  一个基于操作转换的协同编辑系统通常包含两个关键部分: 上层的并发控制算法以及底层的操作转换函数。
  控制算法负责决定何时对哪些操作进行转换,而操作转换函数则决定如何对两个操作进行转换。

  要保证收敛性,底层的操作转换函数需要满足一定的性质,如本文所关心的 CP1 (Convergence Property 1) 性质。
  现有的针对复制列表的操作转换函数通常仅支持简单的插入、删除等基本操作。
  这对于实际应用来说是不够的。
  本文参考 Redis 开源系统中的列表数据结构,
  实现了14种非阻塞列表操作的操作转换函数,
  并使用TLA+工具对其正确性进行了验证。

  本文的主要贡献包括:
  \begin{itemize}
    \item 我们基于操作所涉及的列表范围,将 Redis 列表数据结构所支持的14种非阻塞操作分为三类,
      并针对每一类(尤其是最具广泛性的第三类操作)设计了相应的操作转换函数。
    \item 我们使用 TLA+ 工具验证(三类操作的)操作转换函数的正确性,即是否满足 CP1 性质。
      我们对验证的性能也作了统计与分析。
  \end{itemize}

  % 中文关键词。关键词之间用中文全角分号隔开,末尾无标点符号。
  \keywords{协同编辑系统;操作转换;Redis 列表数据结构; TLA+}
\end{abstract}

%%%%%%%%%%%%%%%%%%%%%%%%%%%%%%%%%%%%%%%%%%%%%%%%%%%%%%%%%%%%%%%%%%%%%%%%%%%%%%%
% 论文的英文摘要
\begin{englishabstract}
Collaborative editing systems allow multiple users to concurrently edit the same document.
For low latency and high availability, such systems often replicate the document at several replicas.
A user can edit the document at a local or nearby replica, 
and the updates are propagated to other replicas asynchronously.
Meanwhile, the documents at different replicas should be kept consistent. 

The replicated list object has been frequently used to model the core functionality 
(e.g., insertion and deletion) of collaborative editing systems.
Convergence is a common specification of a replicated list, a formal requirement of document consistency.
Based on the idea of operational transformation (OT), 
the researchers have proposed several protocols to achieve convergence.
A typical OT-based system consists of two layers: 
the concurrency control algorithm at the top layer and the OT functions at the bottom layer.
The concurrency control algorithm decides the time to transform and the operations to be transformed,
while the OT functions decide the way how a pair of operations are transformed.

To achieve convergence, the OT functions are required to satisfy some properties, 
such as CP1 (Convergence Property 1) we are concerned about in this paper.
Existing OT functions for replicated lists are often limited to the character-wise operations, 
i.e., insertion and deletion.
In this paper, we have designed a complete set of OT functions for 14 non-blocking list operations supported by Redis, 
a well-known open-source distributed key-value store.
We have also verified the correctness (with respect to CP1) of these OT functions using TLA+.

We make two main contributions in this paper:
\begin{itemize}
  \item We divide all the 14 non-blocking operations of Redis list into 3 categories, 
    according to the positions involved. 
    Then we design the OT functions for each category.
    We focus on the third category, consisting of the most general operations.
  \item 
    We verify the correctness of the OT functions of three categories with respect to CP1
    using TLA+.
    The performance of verification is also given.
\end{itemize}

  % 英文关键词。关键词之间用英文半角逗号隔开,末尾无符号。
  \englishkeywords{collaborative editing system, operational transformation, Redis list, TLA+}
\end{englishabstract}
	
	%%%%%%%%%%%%%%%%%%%%%%%%%%%%%%%%%%%%%%%%%%%%%%%%%%%
	% 生成论文目次
	\tableofcontents
	
	%%%%%%%%%%
	% 开始正文部分
	\mainmatter
	
	
	%前言
	\chapter{前言}
\section{应用背景:协同编辑应用}
	\par 协同编辑系统,可以允许不同地点的用户同时编辑同一份文档。为了获得较快的响应和较高的实用性,系统会在不同的地点或设备进行文档的复制。一个用户可以在某个副本上进行文档的编辑,并将做出的修改异步地传递给其他副本。不必要等待服务器处理完再响应用户操作,本地操作可以立即执行。同时系统必须保证编辑的一致性,即在所有用户完成文档的编辑后,所有的副本内容一致。
\begin{figure}
\centering
\includegraphics{figures/timg.jpg}
\caption{系统编辑实例}
\label{fig:graph}
\end{figure}
\section{技术背景:Replicated List 规约及其基于OT的 Replicated List 算法}

\section{本文研究工作: 面向 Redis List的OT函数的设计、验证与实现}
	\par 本次毕业设计的目标是实现Redis List所支持的14种非阻塞操作的OT(Operational Transformation)函数,并且对实现函数的正确性进行验证。阿里云和RedisLab的团队目前都在对Redis List的操作进行开发,Redis List操作的OT函数实现具有应用前景和商业前途。
\section{论文组织}
	\par 本文后续内容组织如下:
	\par 第2章介绍本文的相关工作,包括系统模型和已有的相关OT函数的设计。
	\par 第3章介绍了Redis列表相关的基本命令,并对其进行了分类,然后进行了对应OT函数的设计。
	\par 第4章介绍了TLA+,并使用TLA+完成了对上述设计好的OT函数的验证。
	\par 第5章分析实验结果,并对实验结果进行分析。
	\par 第6章是本论文的结论和以后工作的相关展望。
	%相关工作
	\chapter{相关工作}
\section{OT 函数的性质}
	OT 函数需要满足的性质:
	\begin{itemize}
	  \item Convergence Property 1(CP1): 这是Jupiter 协议正确性的必要条件。对于定义在文档状态S上的给定操作O1和O2,满足CP1等式:$S \circ O1 \circ OT(O2,O1) = S \circ O2 \circ OT(O1,O)$,也就是说在S上按顺序实施O1,OT(O2,O1)操作和实施O2,OT(O1,O2)操作效果相同。
	  \item Convergence Property 2(CP2):对于定义在文档状态S上的给定操作O1和O2,满足CP2等式:$OT(OT(O1,O2),OT(O3,O2)) = OT(OT(O1,O3),OT(O2,O3))$
	  \item Inverse Property 1(IP1)
	  \item Inverse Property 2(IP1)
	  \item Inverse Property 3(IP3)
	\end{itemize}
	
\section{OT函数的设计}
可以设计OT函数,并通过控制算法的调用来保证最终结果的一致性。本文中将Ins(插入),Del(删除),Set(设置)单个元素的简单操作称为第一类操作,将Ins,Del单个区间的操作称为第二类操作,作用于多个区间的操作称为第三类操作(本文中只讨论删除多个区间的操作)。
假设list当前状态为s,那么具体操作的定义如下:
\begin{table}[H]
\centering
\begin{tabular}{|c|c|c|} 
\hline
操作分类 &操作类型 &操作效果\\
\hline
\multirow{3}*{第一类操作}  &$Del(i)$  &删除第i个元素\\ 
\cline{2-3} 
&$Ins(i,x)$ &在位置i插入元素x\\
\cline{2-3}
&$Set(i,x)$ &将位置i上的元素设为x\\
\hline

\multirow{2}*{第二类操作}  &$Ins(p,s)$  &在位置i插入串s\\ 
\cline{2-3} 
&$Del(p,l)$ &从位置p开始删除长度为l的串\\
\hline
第三类操作 &$Del(p_1,l_1;p_2,l_2;...;p_k,l_k)$ &删除区间$[p_1,p_1+l_1],...,[p_k,p_k+l_k]$内的元素\\

\hline
\end{tabular}
\end{table}

现在Ins,Del,Set等简单操作相互之间的OT函数已经设计好,Ins,Del单个区间的OT函数也已经基本上设计完成。
\subsection{第一类 OT 函数的设计}
我们定义第一类函数为第一类操作之间的OT函数,即Ins,Del,Set三种操作之间的OT函数,共有3*3=9个。
这里列出了Del操作对应于Ins,Del,Set的OT函数,全部9个函数的数学公式详见附录。
\begin{equation}
\begin{aligned}
Del \begin{cases}
OT(Del (i), Set (j,x)) =
	{Del(i)}\\
OT(Del (i), Ins (j,x)) =\begin{cases}
	{Del (i+1)}  \quad &i \ge j\\
	{Del (i)}   \quad &i < j\\ \end{cases}\\
OT(del (i), del (j)) =\begin{cases}
	{Del (i-1)} \quad &i > j\\
	{Del (i)} \quad &i < j\\
	{no-op}   \quad &i = j \end{cases}\\
\end{cases}
\end{aligned}
\end{equation}
套用公式解释前言中举例的OT变换,$OT(del (4), del (2)) = del(3)$,$OT(del (2), del (4)) = del(2)$,与例子相符。

\subsection{第二类 OT 函数设计}
我们定义第二类函数为第二类操作之间的OT函数,即Ins,Del两种区间操作之间的OT函数,共2*2=4个。
这里列出了Ins操作对应于Ins,Del的OT函数,全部4个函数的数学公式详见附录。
\begin{equation}
OT(Ins(p1,s1),Ins(p2,s2))= \begin{cases}
Ins(p1,s1) \quad & p1<p2 \\
Ins(p1+ |s2|,s1) \quad & p1>p2 \\
Ins(p1+ |s2|,s1) \quad & p1=p2 \quad pr1<pr2 \\
Ins(p1,s1) \quad & p1=p2 \quad pr1>pr2
 \end{cases}\\
\end{equation}
Ins自身的
\begin{equation}
OT(Ins(p1,s1),Del(p2,l1))= \begin{cases}
Ins(p1,s1) \quad & p1 \le p2 \\
no-op \quad & p2<p1<p2+l1\\
Ins(p1-l1,s1) \quad & p1 \ge p2+l1 \end{cases}\\
\end{equation}

\section{OT 函数的验证}

CP1与CP2性质要求OT函数的设计者考虑所有可能的操作组合情况,
并且,对一种特定的操作组合,还要考虑所有可能的访问位置的组合情况。
OT函数的设计易于出错。
因此,有一系列工作致力于如何验证OT函数的正确性。
Imine 等人~\cite{Imine:TCS06} 基于代数规约框架开发了OT函数自动验证工具 VOTE。
该工具可以自动验证给定的OT函数的正确性或者给出反例。
Imine 等人使用该工具检测了 GROVE~\cite{Ellis:SIGMOD89}, 
Joint Emacs~\cite{Ressel:CSCW96}, REDUCE~\cite{Sun:TOCHI98}, 
CoWord, SAMS~\cite{Molli:CSCW02} 和 So6~\cite{Sun:CSCW04} 协同编辑器。
结果发现,这六个协同编辑器所使用的OT函数都不满足CP2性质,
而且 GROVE 和 Joint Emacs 的OT函数也不满足CP1性质。
Sun 等人~\cite{Sun:CSCW14} 开发了一套验证框架和一个验证工具,
能够穷举所有可能的操作转换情况。
基于此,作者不仅重现了之前已有工作中指出的OT函数设计的错误之处,
而且揭示了所有可能的OT函数的设计错误。
在其后续工作中,Sun 等人~\cite{Sun:CSCW17} 推广了该方法,
用于验证支持字符串插入与字符串删除的列表OT函数的正确性。
在这两份工作中,作者不仅考虑了CP1和CP2性质,
还考虑了OT函数可能需要满足的其它性质。
Randolph 等人~\cite{Randolph:arXiv13} 从理论上考察满足CP1和CP2性质的OT函数的存在性问题。
作者将该问题转化为 (Controller Synthesis) 合成问题,
然后尝试借助形式化验证工具 Uppaal-Tiga~\cite{Cassez:CONCUR05} 自动合成OT函数。
结果表明,在不使用位置(position)之外的信息的情况下,不存在满足CP2性质的OT函数。
Liu 等人~\cite{Liu:FM04} 利用模型检验技术验证了某些OT函数满足CP1性质,
也重现了违反CP2性质的OT函数的设计错误。
此外,作者还第一次验证了一类基于操作上下文 (Operation Context) 概念~\cite{Sun:CSCW06} 的OT控制算法的正确性。
	%OT函数设计
	\chapter{Redis List OT 函数设计}
\section{Redis List API 分类}
\subsection{Redis List API简介}
\par 由于本次毕业设计是针对Redis系统的List命令进行的,首先对Redis中的列表(List)及其相关命令进行一个简要的介绍。
Redis中的列表是字符串列表,按照插入的顺序进行排序,一个列表可以包含的最多元素为$2^{32}-1(4294967295)$个。

\par 列表相关的基本命令共有17个,其中3个为阻塞性操作(即没有元素时会阻塞列表直至等待到有元素可以执行该命令),即其余14种为非阻塞性操作,即下面列出的14个命令。\\

\begin{itemize}
\item LINDEX  key index:通过索引index来获取列表key中的元素
\item LINSERT key BEFORE(AFTER) pivot value:在列表key某个元素pivot的前面(BEFORE)或者后面(AFTER)插入元素value,若元素不在列表中或者列表不存在则不执行任何操作
\item LLEN key:获取列表key的长度,若key不存在则返回0,如果key不为列表类型则出错
\item LPOP key:用于移除并返回列表的第一个元素
\item LPUSH key value1 [value2]:将一个值(value1)或者多个值(value1,value2..)插入到列表key头部,若key不存在则创建一个新列表并执行操作
\item LPUSHX key value:将一个值(value)插入到列表key头部,若列表key不存在则操作无效
\item LRANGE key start stop:返回列表key中指定区间[start,stop]内的元素
\item LREM key count value:根据count的值,移除列表key中与value值相等的元素(即删除count个)
若$count > 0$从表头开始搜索,若$count < 0$从表尾开始搜索,若$count = 0$则移除key中所有与value相等的元素
\item LSET key index value:通过索引index来设置列表key中元素的值为value
\item LTRIM key start stop:对列表key进行修剪,只保留[start,stop]之间的元素,不在该区间的元素全部删除
\item RPOP key:移除并获取列表key中的最后一个元素
\item RPOPLPUSH source destination:移除并获取列表source中的最后一个元素,并添加到另一个列表destination中,返回该列表
\item RPUSH key value1 [value2]:将一个值(value1)或者多个值(value1,value2..)插入到列表key尾部,若key不存在则创建一个新列表并执行操作
\item RPUSHX key value:将一个值(value)插入到列表key尾部,若列表key不存在则操作无效
\end{itemize}

\par 而其中LRANGE,LLEN和LINDEX这三个命令与list的内容修改无关,因此在本文中不考虑这两个命令的相关OT操作。本文中只考虑剩余这11个命令的OT函数的设计和验证。
\subsection{Redis List API分类}
\par 经过分析,这12个命令可以根据操作类型和作用范围分为以下这三类。
\begin{itemize}
\item 单个元素的删除、修改、插入:$ Ins(pos,ele),Del(pos),Set(pos,ele)$\\
\item 单个区间的删除、插入:$Ins(pos,str),Del(pos,len)$\\
\item 多个区间的删除:$ Del(pos1,len1;pos2,len2;...;posk,lenk) $\\
\end{itemize}
\par 具体来说,就是
\begin{itemize}
\item 第一类命令:
	\begin{itemize}
	\item $LPUSHX -> Ins(0,ele)$
	\item $RPUSHX -> Ins(len,ele)$
	\item $LINSERT -> Ins(pos,ele)$
	\item $LPOP -> Del(0)$
	\item $RPOP -> Del(len-1)$
	\item $RPOPLPUSH  -> Del(len-1)$
	\item $LSET -> Set(pos,ele)$
	\end{itemize}
\item 第二类命令:
	\begin{itemize}
	\item $LPUSH -> Ins(0,str)$
	\item $RPUSH -> Ins(len,str)$
	\end{itemize}
\item 第三类命令:
	\begin{itemize}
	\item $LTRIM -> Del(0,pos1-1;pos2+1,len-pos2-1)$
	\item $LREM ->  Del(pos1,len1;pos2,len2;...;posk,lenk)$
	\end{itemize}
\end{itemize}

\section{第一类 OT 函数的设计}
\par 我们定义第一类函数为第一类命令之间的OT函数,即Ins,Del,Set三种操作之间的OT函数,共有3*3=9个。
\begin{equation}
\begin{aligned}
Set \begin{cases}
OT(set (i,x), set (j,y)) =\begin{cases}
    no-op \quad &pr1 > pr2 \quad i=j\\
	{set (i,x)} \quad &else \end{cases} \\ 
OT(Set(i,x),Ins(j,y))=\begin{cases}
{Set(i,x)}  \quad &i<j\\
{Set(i+1,x)} \quad  &i\ge j \end{cases} \\
OT(Set(i,x),Del(j))=\begin{cases}
{Set(i,x)} \quad &i<j\\
{no-op} \quad & i=j\\
{Set(i-1,x)} \quad &i>j \end{cases} \\
\end{cases}
\end{aligned}
\end{equation}
\begin{equation}
\begin{aligned}
Ins \begin{cases}
OT(Ins(i,x), set (j,y)) =
{Ins(i,x)}\\
OT(ins (i,x), ins (j,y)) =\begin{cases}
	{ins(i+1, x)}   \quad & i > j\\
	{ins(i, x)}    \quad & i < j\\
	{ins(i+1, x)}   \quad  & i = j \quad pr1 < pr2\\
	{ins(i, x)}   \quad  & i = j \quad pr1 > pr2 \end{cases} \\
OT(Ins(i,x),Del(j))=\begin{cases}
{Ins(i,x)}  \quad i \le j\\
{Ins(i-1,x)} \quad i>j \end{cases}\\
\end{cases}
\end{aligned}
\end{equation}


\begin{equation}
\begin{aligned}
Del \begin{cases}
OT(Del (i), Set (j,x)) =
	{Del(i)}\\
OT(Del (i), Ins (j,x)) =\begin{cases}
	{Del (i+1)}  \quad &i \ge j\\
	{Del (i)}   \quad &i < j\\ \end{cases}\\
OT(del (i), del (j)) =\begin{cases}
	{Del (i-1)} \quad &i > j\\
	{Del (i)} \quad &i < j\\
	{no-op}   \quad &i = j \end{cases}\\
\end{cases}
\end{aligned}
\end{equation}

\section{第二类 OT 函数设计}
我们定义第二类函数为第二类命令之间的OT函数,即Ins,Del两种命令之间的OT函数,共2*2=4个。
\begin{equation}
OT(Ins(p1,s1),Ins(p1,s2))= \begin{cases}
Ins(p1,s1) \quad & p1<p2 \\
Ins(p1+ |s2|,s1) \quad & p1>p2 \\
Ins(p1+ |s2|,s1) \quad & p1=p2 \quad pr1<pr2 \\
Ins(p1,s1) \quad & p1=p2 \quad pr1>pr2
 \end{cases}\\
\end{equation}

\begin{equation}
OT(Ins(p1,s1),Del(p2,l1))= \begin{cases}
Ins(p1,s1) \quad & p1 \le p2 \\
no-op \quad & p2<p1<p2+l1\\
Ins(p1-l1,s1) \quad & p1 \ge p2+l1 \end{cases}\\
\end{equation}

\begin{equation}
OT(Del(p1,l1),Ins(p2,s1))= \begin{cases}
Del(p1,l1) \quad & p1 + l1 \le p2 \\
Del(p1,l1+|s1|) \quad & p1<p2<p1+l1 \\
Ins(p1+ |s1|,l1) \quad & p1 \ge p2 \end{cases}\\
\end{equation}

\begin{equation}
OT(Del(p1,l1),Del(p2,l2))= \begin{cases}
Del(p1,l1) \quad & p1<p2 \quad p1+l1 \le p2 \\
Del(p1,p2-p1) \quad & p1<p2 \quad p2<p1+l1 \le p2+l2\\
Del(p1,l1-l2) \quad & p1<p2 \quad p2+l2<p1+l1\\
no-op \quad & p2 \le p1 < p2+l2 \quad p1+l1 \le p2+l2\\
Del(p2,p1+l1-p2-l2) \quad & p2 \le p1 <p2+l2 \quad  p1+l1>p2+l2\\
Del(p1-l2,l1) \quad & p1 \ge p2+l2  \end{cases}\\
\end{equation}

\section{第三类 OT 函数设计}
\par 我们定义第三类函数为第三类的Del命令与第二类命令中的Ins操作之间的OT函数,以及第三类Del命令自身的OT函数,共2+1=3个。
\par 首先考虑第三类的Del命令与第二类命令中的Ins操作之间的OT函数,显然,我们需要按Ins操作的插入位置和Del操作的删除区间之间的关系进行分类,进行函数的设计。
\par Ins操作对于Del操作的转换,如果是插入位置位于删除区间中,则Ins操作转换为NOP,否则插入的位置要减去删除操作在该位置之前删除区间的总长度。\\
$OT(Ins(p_{k+1},s_{k+1}),Del(p_1,l_1;p_2,l_2;...;p_k,l_k))$\\
\begin{equation}
= \begin{cases}
Ins(p_{k+1},s_{k+1}) \quad & p_{k+1} \le p_1 \\
no-op \quad & p_i<p_{k+1}<p_i+l_i\\
Ins(p_{k+1}-l_1-l_2-...-l_i,s_{k+1}) \quad & p_i+l_i \le p_{k+1} \le p_{i+1}\\
Ins(p_{k+1}-l_1-l_2-...-l_k,s_{k+1}) \quad & p_{k+1} \ge pk+lk \end{cases}\\
\end{equation}

\par Del操作对于Del操作的转换,如果是插入位置位于删除区间中,则该区间删除长度增加$|s|$,之前的区间不变,之后的区间都向后移动$|s|$单位长度
\par 如果插入位置不在删除区间中,那么在插入位置之前的区间不变,之后的区间都向后移动$|s|$单位长度\\
$OT(Del(p_1,l_1;p_2,l_2;...;p_k,l_k),Ins(p_{k+1},s_{k+1}))$\\
\begin{equation}
= \begin{cases}
Del(p_1+|s_{k+1}|,l_1;p_2+|s_{k+1}|,l_2;...;p_i,l_i;p_{i+1}+|s_{k+1}|,l_{i+1};...;p_k+|s_{k+1}|,l_k) & \quad p_{k+1} \le p_1 \\
Del(p_1,l_1;p_2,l_2;...;p_{i-1},l_{i-1};p_i,l_i+|s_{k+1}|;p_{i+1}+|s_{k+1}|,l_{i+1};...;p_k+|s_{k+1}|,l_k) &\quad p_i<p_{k+1} < pi+li \\
Del(p_1,l_1;p_2,l_2;...;p_i,l_i;p_{i+1}+|s_{k+1}|,l_{i+1};...;p_k+|s_{k+1}|,l_k) & \quad p_i+l_i \le p_{k+1} \le p_{i+1}\\
Del(p_1,l_1;p_2,l_2;...;p_k,l_k) &\quad P_k+l_k \le p_{k+1} \\
 \end{cases}\\
\end{equation}

\par Del操作自身的转换时最为复杂的,一开始想要将要转换的Del操作中所有的区间一起转换,发现这样不仅做起来难度很大,而且写公式和用代码表达都很容易出错,但如果是将每个区间都用某个公式来转换,然后将转换后的新区间合并为新的删除操作,就可以方便的实现第三类Del操作自身的转换了。
\par 转变思路后,第三类Del自身的转换就可以用第二类Del操作对于第三类Del操作的转换来代替了,减少了公式的复杂度,同时也增加了可读性。
\par 与前面的设计类似,由删除区间与删除区间之间的位置关系进行函数的设计。\\
\newpage
$OT(Del(p_{k+1},l_{k+1}),Del(p_1,l_1;p_2,l_2;...;p_k,l_k))$\\
\begin{equation}
= \begin{cases}
Del(p_{k+1},l_{k+1}) \quad &p_{k+1} < p_1 \quad p_{k+1}+l_{k+1} \le p_1 \\
Del(p_{k+1},p_j-l_1-l_2-...-l_{j-1}-p_{k+1}) \quad &p_{k+1} < p_1 \quad p_j < p_{k+1}+l_{k+1} \le p_j+l_j \\
Del(p_{k+1},l_{k+1}-l_1-l_2-...-l_j) \quad &p_{k+1} < p_1 \quad p_j+l_j < p_{k+1}+l_{k+1} \le p_{j+1} \\
Del(p_{k+1},l_{k+1}-l_1-l_2-...-l_k) \quad &p_{k+1} < p_1 \quad p_{k+1}+l_{k+1} > P_k+l_k  \\

Del(p_i-l_1-l_2-...-l_{i-1},p_j-p_i-l_i-l_{i+1}...-l_{j-1})        \quad &p_i \le p_{k+1} < p_i+l_i \quad \\ &p_j < p_{k+1}+l_{k+1} \le p_j+l_j \\
Del(p_i-l_1-l_2-...-l_{i-1},p_{k+1}+l_{k+1}-p_i-l_i-l_{i+1}-...-l_j) \quad &p_i \le p_{k+1} < p_i+l_i \quad \\& p_j+l_j < p_{k+1}+l_{k+1} \le p_{j+1} \\
Del(p_i-l_1-l_2-...-l_{i-1},p_{k+1}+l_{k+1}-p_i-l_i-l_{i+1}-...-l_k) \quad &p_i \le p_{k+1} < p_i+l_i \quad \\& p_{k+1}+l_{k+1} > P_k+l_k  \\


Del(p_{k+1}-l_1-l_2-...-l_{i-1},p_j-p_{k+1}-l_{i+1}-l_{i+2}-...-l_{j-1})    \quad &p_i+l_i \le p_{k+1} < p_{i+1} \quad \\& p_j < p_{k+1}+l_{k+1} \le p_j+l_j \\
Del(p_{k+1}-l_1-l_2-...-l_{i-1},l_{k+1}-l_{i+1}-l_{i+2}-...-l_j)    \quad &p_i+l_i \le p_{k+1} < p_{i+1} \quad \\& p_j+l_j < p_{k+1}+l_{k+1} \le p_{j+1} \\
Del(p_{k+1}-l_1-l_2-...-l_{i-1},l_{k+1}-l_{i+1}-l_{i+2}-...-l_k)    \quad &p_i+l_i \le p_{k+1} < p_{i+1} \quad \\& p_{k+1}+l_{k+1} > P_k+l_k  \\
Del(p_{k+1}-p_1-p_2...-p_k,l_{k+1}) \quad &p_{k+1} \ge p_k+l_k
\\&(i \ge j)
 \end{cases}\\
\end{equation}
\section{剩余 OT 函数设计}
\par 显然,第三类的del命令可以涵盖第一类和第二类的del命令,第二类的ins命令可以涵盖第一类的ins命令,因此我们最后只要考虑set(pos,ele)命令和第二类的ins操作、第三类的del操作之间的OT函数关系,即可覆盖完全所有的OT函数设计。共有2+2=4个函数。

\par Ins操作对于Set操作的转换就是其本身。\\
\begin{equation}
OT(Ins(i,s),Set(j,x))= Ins(i,s)\\
\end{equation}

\par Set操作对于Ins操作的转换,如果插入位置在Set位置之前则Set位置要增加$|s|$单位长度,否则Set操作不变。\\
\begin{equation}
OT(Set(i,x),Ins(j,s))= \begin{cases}
{Set(i,x)}  \quad & i<j \\
{Set(i+|s|,x)} \quad  &i \ge j \end{cases} \\
\end{equation}

\par Del操作对于Set操作的转换就是其本身。\\
$OT(Del(p_1,l_1;p_2,l_2;...;p_k,l_k),Set(p_{k+1},x)) = Del(p_1,l_1;p_2,l_2;...;p_k,l_k)$\\

\par Set操作对于Del操作的转换,如果是Set位置位于删除区间中,则Ins操作转换为NOP,否则插入的位置要减去删除操作在该位置之前删除区间的总长度。\\
$OT(Set(p_{k+1},x),Del(p_1,l_1;p_2,l_2;...;p_k,l_k))$
\begin{equation}
= \begin{cases}
Ins(p_{k+1},s_{k+1}) \quad & p_{k+1} < p_1 \\
no-op \quad & p_i \le p_{k+1} < p_i+l_i\\
Set(p_{k+1}-l_1-l_2-...-l_i,x) \quad & p_i+l_i \le p_{k+1} < p_{i+1}\\
Set(p_{k+1}-l_1-l_2-...-l_k,x) \quad & p_{k+1} \ge pk+lk \end{cases}\\
\end{equation}
	%OT函数验证
	\chapter{基于 TLA+ 的 OT函数验证} 
\section{TLA+ 简介}
\par TLA+是一种形式化的规范语言。它是一种设计系统和算法的工具,并且用来验证这些系统有没有关键错误。
\par 正确性,是一个系统最为重要的性质,同时,正确性是比较难以证明的,特别是并发系统的正确性,因为存在着数目众多的状态变化,而TLA+可以将系统的行为或者状态抽象为时态逻辑,即系统的行为或者状态会随着时间反生变化,然后通过一些数学分析的方法,来判断系统是否正确。
\par TLA+并不同于一般传统意义上的编程语言,更类似于一种数学语言,因为其语法大部分来自于实际的数理逻辑。
\par TLA+提供了工具集TLAToolbox/TLC,同时还可以使用TLA+的语法糖PLUSCAL来完成代码的编写,由于本文中并未涉及,在此不做展开。
\begin{figure}
\centering
\includegraphics{figures/module.jpg}
\caption{TLA+编码模板}
\label{fig:graph}
\end{figure}

\section{使用 TLA+ 描述 OT 函数}
\subsection{调用模块的简单介绍}
\par 在本次实验中,我们引入了TLA+ 中的Sequences模块,将List表示为一个Sequecnce,并且使用了如下的API:\\
\begin{tabular}{ccc}
\hline
operator& operation& example \\
\hline  
 Head& First element &Head(<<1, 2>>) = 1\\
 Tail& Sequence aside from head &Tail(<<1, 2>>) = <<2>>\\
 Append& Add element to end of sequence &Append(<<1>>, 2) = <<1, 2>>\\ 
 Len& Length of sequence &Len(<<1, 2>>) = 2\\
\hline % 
\end{tabular}
 	 	
\subsection{OT函数设计}
\paragraph{第一、二类函数的表示}
\paragraph{第三、四类函数的表示}
\paragraph{命令执行的表示}

\section{正确性验证}
\par 在第二章中,我们提到过CP1和CP2两个协议,在这里我们只验证CP1性质的正确性。即同一个List经过OT(OP2,OP1),OP1或者OT(OP1,OP2),OP1这两种操作序列后,最终的结果是一致的。
\par 用TLA+来描述就是:
\par $apply(apply(list,op1),Xform(op2, op1)) = apply(apply(list,op2),Xform(op1, op2)) $
\par 只要对于任意的两个操作,这个等式都成立的话,那么CP1正确性即可得到验证。
	%未来工作
	\chapter{结论与未来工作}
\par Redis List命令的OT函数设计经过验证是成功的。
\section{工作总结}

\section{研究展望}

	%参考文献
	\bibliography{thesis}
	%致谢
	\begin{acknowledgement}
  本文的工作是在魏恒峰老师的悉心指导下完成的。
  魏老师严谨的治学态度和认真的工作态度给了我极大的影响。
  在相关知识的学习、实验过程以及论文的写作过程中,魏老师也给了我莫大的帮助。
  在此衷心感谢魏老师这半年的关心和指导。

感谢黄宇老师和马晓星老师,他们在我的学习和实验过程中也给出了很多宝贵的指导和建议。

感谢我的同学们,很幸运能与易星辰、王芷芙、唐瑞泽等同学在 TLA+ 讨论班中共同学习和进步。
感谢黄奕学长,张宇奇学长在我实验过程中给与的帮助和指导。
感谢缪娅,谢谢你的理解和陪伴。

感谢我的父母,谢谢你们一直关心着我,一直给我无私的爱。
你们的支持是我顺利完成学业的最大动力。
\end{acknowledgement}
	%附录
	\appendix
\chapter{附录} 
\section{OT函数的设计}
\subsection{第一类 OT 函数的设计}
第一类函数为第一类命令之间的OT函数,即Ins,Del,Set三种命令之间的OT函数,共有$3*3=9$个。
\begin{equation}
\begin{aligned}
Set \begin{cases}
OT(set (i,x), set (j,y)) =\begin{cases}
    no-op \quad &pr1 > pr2 \quad i=j\\
	{set (i,x)} \quad &else \end{cases} \\ 
OT(Set(i,x),Ins(j,y))=\begin{cases}
{Set(i,x)}  \quad &i<j\\
{Set(i+1,x)} \quad  &i\ge j \end{cases} \\
OT(Set(i,x),Del(j))=\begin{cases}
{Set(i,x)} \quad &i<j\\
{no-op} \quad & i=j\\
{Set(i-1,x)} \quad &i>j \end{cases} \\
\end{cases}
\end{aligned}
\end{equation}
\begin{equation}
\begin{aligned}
Ins \begin{cases}
OT(Ins(i,x), set (j,y)) =
{Ins(i,x)}\\
OT(ins (i,x), ins (j,y)) =\begin{cases}
	{ins(i+1, x)}   \quad & i > j\\
	{ins(i, x)}    \quad & i < j\\
	{ins(i+1, x)}   \quad  & i = j \quad pr1 < pr2\\
	{ins(i, x)}   \quad  & i = j \quad pr1 > pr2 \end{cases} \\
OT(Ins(i,x),Del(j))=\begin{cases}
{Ins(i,x)}  \quad i \le j\\
{Ins(i-1,x)} \quad i>j \end{cases}\\
\end{cases}
\end{aligned}
\end{equation}


\begin{equation}
\begin{aligned}
Del \begin{cases}
OT(Del (i), Set (j,x)) =
	{Del(i)}\\
OT(Del (i), Ins (j,x)) =\begin{cases}
	{Del (i+1)}  \quad &i \ge j\\
	{Del (i)}   \quad &i < j\\ \end{cases}\\
OT(del (i), del (j)) =\begin{cases}
	{Del (i-1)} \quad &i > j\\
	{Del (i)} \quad &i < j\\
	{no-op}   \quad &i = j \end{cases}\\
\end{cases}
\end{aligned}
\end{equation}

\subsection{第二类 OT 函数的设计}
第二类函数为第二类命令之间的OT函数,即Ins,Del两种命令之间的OT函数,共$2*2=4$个。
\begin{equation}
OT(Ins(p1,s1),Ins(p1,s2))= \begin{cases}
Ins(p1,s1) \quad & p1<p2 \\
Ins(p1+ |s2|,s1) \quad & p1>p2 \\
Ins(p1+ |s2|,s1) \quad & p1=p2 \quad pr1<pr2 \\
Ins(p1,s1) \quad & p1=p2 \quad pr1>pr2
 \end{cases}\\
\end{equation}

\begin{equation}
OT(Ins(p1,s1),Del(p2,l1))= \begin{cases}
Ins(p1,s1) \quad & p1 \le p2 \\
no-op \quad & p2<p1<p2+l1\\
Ins(p1-l1,s1) \quad & p1 \ge p2+l1 \end{cases}\\
\end{equation}

\begin{equation}
OT(Del(p1,l1),Ins(p2,s1))= \begin{cases}
Del(p1,l1) \quad & p1 + l1 \le p2 \\
Del(p1,l1+|s1|) \quad & p1<p2<p1+l1 \\
Ins(p1+ |s1|,l1) \quad & p1 \ge p2 \end{cases}\\
\end{equation}

\begin{equation}
OT(Del(p1,l1),Del(p2,l2))= \begin{cases}
Del(p1,l1) \quad & p1<p2 \quad p1+l1 \le p2 \\
Del(p1,p2-p1) \quad & p1<p2 \quad p2<p1+l1 \le p2+l2\\
Del(p1,l1-l2) \quad & p1<p2 \quad p2+l2<p1+l1\\
no-op \quad & p2 \le p1 < p2+l2 \quad p1+l1 \le p2+l2\\
Del(p2,p1+l1-p2-l2) \quad & p2 \le p1 <p2+l2 \quad  p1+l1>p2+l2\\
Del(p1-l2,l1) \quad & p1 \ge p2+l2  \end{cases}\\
\end{equation}

\end{document}